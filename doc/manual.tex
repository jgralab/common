\documentclass[a4paper,twoside,11pt,bibtotoc]{article}
\usepackage[utf8]{inputenc}
\usepackage[T1]{fontenc}
\usepackage{courier}
\usepackage{palatino}
\usepackage{geometry}
%\usepackage{savesym}
\usepackage{amssymb}
%\usepackage{amsmath}
\usepackage{oz}
\usepackage{etex}
\usepackage{listings}
%\usepackage{url}
\usepackage{setspace}
%\usepackage{tabularx}
%\usepackage{multirow}
\usepackage{fancyhdr}
\usepackage{color}
\usepackage{graphicx}
\usepackage{tikz}
\usepackage{fancybox}
\usepackage{framed}
\usepackage{enumerate}
\usepackage{paralist}
\usepackage{mdwlist}
\usepackage[pdftex,plainpages=false]{hyperref}

%\overlinesegment
%\underlinesegment

% Version information
\newcommand{\version}{Version 0.1 -- \today}

\newcommand{\makeisttitle}[2] % 1st argument author; 2nd argument 
{
\unitlength1cm
\begin{picture}(0,1)\put(0,1){\makebox(0,0)[tl]{\includegraphics[height=14mm]{images/ist-logo-en.pdf}}}\end{picture}
\hfill
\begin{picture}(0,0)\put(0,1){\makebox(0,0)[tr]{\includegraphics[height=18mm]{images/uko-logo-fb4.pdf}}}\end{picture}
\begin{center}
\vspace*{3cm}
{\Large #1}\\
\vspace*{0,5cm}
#2
\end{center}}

% special commands
\newcommand{\rselem}[1]{\textsf{#1}}
\newcommand{\discuss}[1]
{
%   \rule{\textwidth}{0.4pt}
% 
%   \begin{footnotesize}
%    \textit{#1}
%   \end{footnotesize}
% 
%   \rule{\textwidth}{0.4pt}
}

% layout
\geometry{verbose, a4paper, tmargin=3.5cm, bmargin=3cm, lmargin=3cm, rmargin=2cm, headheight=2cm, headsep=1.5cm, footskip=1.5cm}

\newcommand{\setindents}{
\setlength\parskip{\medskipamount} %
\setlength\parindent{0pt} %
\onehalfspacing %
}
\setindents

%\setcounter{tocdepth}{2}


% headings
\pagestyle{fancy}
\renewcommand{\headrulewidth}{0pt} 
\renewcommand{\footrulewidth}{0pt}

\newcommand{\setistheader}[1]{
\chead{#1}
\rhead{}
\lhead{}
}

% listings
\lstdefinelanguage{tg}{
 morekeywords={from, to, abstract, VertexClass, EdgeClass, AggregationClass, CompositionClass, GraphClass, Schema, RecordDomain, EnumDomain, String, List, Set, role, Integer, Boolean, Double},
 sensitive=true,
 morecomment=[l]{//},
 morestring=[b]",
}

\lstdefinelanguage{ebnf}{
 morekeywords={},
 sensitive=true,
 string=[b][\bf\ttfamily]",
}

\lstset{language = java, numbers = left, basicstyle = \small\ttfamily, numberstyle = \tiny, columns = fixed, breaklines = true, breakatwhitespace = true, backgroundcolor = \color{LighterGray}, frame = lines, tabsize = 4}

% paragraphs
\clubpenalty = 10000
\widowpenalty = 10000

% bibtex definitions for geralpha style
% \renewcommand{\refname}{References} 
\newcommand{\btxandlong}{and}
\newcommand{\Btxinlong}{in}
\newcommand{\Btxtechreplong}{Technical report}
\newcommand{\btxmastthesis}{master thesis}

%colors for listings
\definecolor{LightGray}{rgb}{0.8,0.8,0.8}
\definecolor{LighterGray}{rgb}{0.9,0.9,0.9} 

% the topic box command creates a labled box with content at the margin of a page.
% Parameters: 1. topic label, 2. color, 3. content text
\newcommand{\topicbox}[3]{%
\marginpar{
  \fontsize{0.4em}{0em} \textbf{#1}\\
  \begin{tikzpicture}
    \tikzstyle{every node}=[fill=#2!20,rounded corners, draw=#2!50, text width=1.5cm,text centered]
    \node[] {\tiny #3};
  \end{tikzpicture}
}%
}%

\newcommand{\inlineboxsep}{\setlength{\fboxsep}{2pt}}


\newenvironment{normalframe}[1][\textwidth]
{\setlength{\fboxsep}{3mm}\begin{Sbox}\begin{minipage}{#1}}%
{\end{minipage}\end{Sbox}\fbox{\TheSbox}}

% \newenvironment{fminipage}%
% {\begin{Sbox}\begin{minipage}}%
% {\end{minipage}\end{Sbox}\fbox{\TheSbox}}

\newenvironment{roundframe}[1][\textwidth]
{\setlength{\fboxsep}{3mm}\begin{Sbox}\begin{minipage}{#1}}%
{\end{minipage}\end{Sbox}\ovalbox{\cornersize{1}\TheSbox}}

% hide topic boxes
% \renewcommand{\topicbox}[3]{}

% note
\newcommand{\Note}[2][?]{%
  \topicbox{Note from #1}{orange}{#2}%
}

% todo
\newcommand{\todo}[2][?]{%
  \topicbox{Todo for #1}{red}{#2}%
}



% hide topic boxes
% \renewcommand{\topicbox}[3]{}


\setistheader{The common ant script}
\parindent0pt



\begin{document}
\sloppy

\begin{titlepage}

\makeisttitle{The common ant script \par (\version)}{Sascha Strau\ss \par
strauss@uni-koblenz.de}


\end{titlepage}

%\cleardoublepage

%\tableofcontents

%\cleardoublepage

\section{Introduction}
The common ant script serves as general build file for projects depending on JGraLab.
It provides all common features an ant script for such projects requires.
An actual ant script for such a project (in the following called \emph{specific ant script}) is derived from the common ant script using ants import mechanism.
In most cases, the specific ant script only needs to set several properties.
For some projects, it requires more adjustments.
However, these adjustments are still less complex.

This document describes the properties that are supported by the common ant script and the targets that can be used in all specific ant scripts.
It also describes how specific ant scripts can be customized and when this is feasible.

\section{Properties}
\label{sec:properties}
This section describes general rules for properties and introduces the general properties that are not target specific.
Target specific properties are introduced in section \ref{sec:targets}.

\subsection{Overview of general properties}
The following list gives an overview of the general properties.
\begin{description*}
	\item[projectname] contains the name of the project. By convention, this must be identical to the project directory and may only contain lowercase letters.\par The default value is empty.
	\item[basePackage] contains the name of the base package.\par The default value is \texttt{"de.uni\_koblenz.\$\{projectname\}"}.
	\item[basePackagePath] contains the path to the base package.\par The default value is \texttt{"de/uni\_koblenz/\$\{projectname\}"}.
	\item[main] contains the name of the main class (the class whose main method is executed, if the jar file is called directly).\par The default value is empty, so it has to be overridden.
	\item[main.fq] contains the fully qualified name of the main class.\par The default value is \texttt{"\$\{basePackage\}.\$\{main\}"}.
	\item[maxmemsize] contains the maximum amount of memory for tasks that use a forked VM.\par The default value is \texttt{"512M"}.
	\item[minmemsize] contains the minimum amount of memory for tasks that use a forked VM.\par The default value is \texttt{"256M"}.
\end{description*}	

The following list shows all properties that contain directory information.
By convention, the default values of these properties should not be overridden.
\begin{description*}
	\item[project.dir] contains the relative path to the project directory.\par The default value is \texttt{"../\$\{projectname\}"}.
	\item[src.dir] contains the relative path to the source directory.\par The default value is \texttt{"\$\{project.dir\}/src"}.
	\item[build.dir] contains the relative path to the build directory.\par The default value is \texttt{"\$\{project.dir\}/build"}.
	\item[classes.dir] contains the relative path to the compiled Java classes.\par The default value is \texttt{"\$\{build.dir\}/classes"}.
	\item[common.dir] contains the relative path to the directory of the project \texttt{common}.\par The default value is \texttt{"../common"}.
\end{description*}
There are further target specific properties with directory information that can be found in section \ref{sec:targets}.

\subsection{The location of property definitions}
Properties in ant are comparable to constants.
Once set, they are immutable.
However, it is possible to override properties in specific ant scripts.
Properties for overriding default values have to be defined before the import clause for the common ant script.

Properties that are used as variables in other property declarations have to be defined before they are used.
This means, custom properties, that use standard properties from the common ant script, have to be defined after the import clause.

Please note, both rules imply that defining properties overriding default values while using other properties with default values is only possible if the used default values are explicitly set before the import clause.
This relation is transitive, so the described scenario should be avoided.

\section{Classpath}
\label{sec:classpath}
The common ant script provides two classpaths.
These classpaths require the following properties.

\begin{description*}
	\item[comlib.dir] contains the relative path to the common libraries.\par The default value is \texttt{"\$\{common.dir\}/lib"}.
	\item[lib.dir] contains the relative path to the project specific libraries.\par The default value is \texttt{"\$\{project.dir\}/lib"}.
	\item[jgralab.location] contains the location of JGraLab's jar file.\par The default value is \texttt{"../jgralab/build/jar/jgralab.jar"}.
	\item[ist\_utilities.location] contains the location of the jar file including the IST utilities.\par The default value is \texttt{"\$\{comlib.dir\}/ist\_utilities/ist\_utilities.jar"}.
\end{description*}

The first and more important classpath is called \texttt{classpath} and contains all important libraries that are required for compiling and using the project.
It includes all compiled java classes, all project specific libraries, the IST utilities and JGraLab.
It also includes the path \texttt{classpathExtension} which has to be defined in specific ant scripts.
Unlike overriding property definitions, this path has to be defined after the import statement.
If libraries from the common library directory are required, these have to be added to \texttt{classpathExtensions}.
If no further entries are required, \texttt{classpathExtensions} has to be empty.

Listing \ref{lst:cp_add} shows an example on how common libraries should be added to the classpath.

\begin{lstlisting}[caption=Adding common libraries to the classpath,label=lst:cp_add,float=!ht,language=ant]
	<path id="classpathExtension">
		<fileset dir="${comlib.dir}/apache/commons/cli/" includes="**/*.jar" />
		<fileset dir="${comlib.dir}/jdom/" includes="**/*.jar" />
	</path>
\end{lstlisting}

Please note that the jar files are not added directly to the classpath, but all jar files in the containing directory.
This is done, because the name of jar files might change due to a more recent version.

The second classpath is called \texttt{testclasspath} and contains \texttt{classpath}, the compiled test cases and Junit's jar file for executing the test cases.
This classpath cannot be directly extended.

\section{Targets}
\label{sec:targets}
This section describes the targets provided by the common ant task.
%First all targets for building the project are introduced.
%Then all targets for cleaning the projects are shown.
%Finally all targets for additional processes are described.

\subsection{Targets for compiling and generating}
\label{sec:creating}
Here all targets that are used for compiling source code and for generating other artifacts are introduced.
The most important target is \texttt{build}, which is also the default target for every specific ant script.
The target \texttt{build} first compiles other required projects.

\subsubsection{Targets for building required projects}
\label{sec:required}
All projects that use the common ant script automatically depend on JGraLab and on the IST utilities.
The common ant script provides targets that call the ant scripts of these projects automatically.
The target for building JGraLab is called \texttt{jgralab}.
The target for building the IST utilities is called \texttt{ist\_utilities}.
The jar files of these projects are only built, if they do not already exist.

After these targets have been executed, the build target executes the target \texttt{clean} (see section \ref{sec:cleaning}) for removing most generated files that might have been created by a previous run of the ant script.
Then the target \texttt{compile} is invoked (see section \ref{sec:compile}).
This target compiles the actual source code of the project.
Since most projects require a custom graph schema, targets for creating the Java files, that represent this schema, have to be generated.

\subsubsection{Schema related targets}
A schema can either be specified by exporting an RSA model to an xmi file or by providing a tg file.
JGraLab can convert xmi files to tg files.
In both cases, a tg file is used for generating Java source files representing the schema.

The target for generating Java source files from tg files is called \texttt{generateschema}.
It is controlled by the following parameters.

\begin{description*}
	\item[schema.file] contains the relative path to the tg file.\par By default this property is unset. Only if it is set to a value, the target \texttt{generateschema} is actually executed.
	\item[schema.location] contains the relative path to the source directory, where the generated Java files should be stored.\par The default value is \texttt{"\$\{src.dir\}"}.
	\item[schema.implementationMode] contains a comma separated list of implementation modes that should be generated.\par The default value is \texttt{"standard"}. E.g. if additional transaction support is required, it should be overridden with the value \texttt{"standard,transaction"}. Valid values may contain (so-far) \texttt{standard}, \texttt{transaction}, \texttt{savemem} and \texttt{database}.
	\item[schema.withoutTypes] corresponds to the flag \texttt{-w} in \texttt{TgSchema2Java}.\par The default value is \texttt{"false"}.
	\item[schema.subtypeFlag] corresponds to the flag \texttt{-f} in \texttt{TgSchema2Java}.\par The default value is \texttt{"false"}.
\end{description*}

The target for converting an exported xmi file to tg is called \texttt{convertschema}.
If an xmi file is specified, this target is executed first.
It is controlled by the following parameters.

\begin{description*}
	\item[xmi.schema.file] contains the relative path to the xmi file.\par By default this property is unset. Only if it is set to a value, the target \texttt{convertschema} is actually executed.
	\item[rsa2tg.f] corresponds to the flag \texttt{f} in \texttt{Rsa2Tg}.\par The default value is \texttt{"false"}.
	\item[rsa2tg.u] corresponds to the flag \texttt{u} in \texttt{Rsa2Tg}.\par The default value is \texttt{"false"}.
	\item[rsa2tg.n] corresponds to the flag \texttt{n} in \texttt{Rsa2Tg}.\par The default value is \texttt{"false"}.
\end{description*}

If the target \texttt{convertschema} is used, also the property \textbf{schema.file} has to be set.
Otherwise the target \texttt{generateschema} would not be executed and the build would most likely fail.

After the schema files have been generated, the target \texttt{compile} can actually compile the source files.
\subsubsection{The target \texttt{compile}}
\label{sec:compile}
The target \texttt{compile} is controlled by the following parameters.

\begin{description*}
	\item[compileincludes] contains a list of ant patterns that specify which source files are being included in the compile process.\par The default value is empty, which means, everything is included.
	\item[compileexcludes] contains a list of ant patterns that specify which source files are being excluded from the compile process.\par The default value is empty, which means, nothing is excluded.
	\item[javac.targetVM] contains the compatibility level of the class files.\par The default value is \texttt{"1.6"}.
	\item[debug] contains information about the debug level of the compiled class files.\par The default value is \texttt{"false"}.
	\item[debuglevel] contains a comma separated list of which debug information will be included, if \textbf{debug} is set to \texttt{"true"}.\par The default value is \texttt{"lines"}. Valid values may contain \texttt{lines}, \texttt{vars} and \texttt{source}.
\end{description*}

After the compilation of the source code is done, the build target executes the targets for creating the jar file.

\subsubsection{Targets for creating the jar file}
The common ant script is designed for creating standalone jar files.
For achieving this, all required libraries are extracted and put into the project's jar file.
The target \texttt{unjar} extracts all required libraries to a temporary folder.

The target \texttt{unjar} by default extracts the file \texttt{ist\_utilities.jar} and all project specific libraries.
It is controlled by the following properties.
\begin{description*}
	\item[tmp.dir] contains the relative path to a non-existing directory that will be created, will serve as temporary directory and will be deleted during the build process.\par The default value is \texttt{"\$\{build.dir\}/tmp"}.
	\item[unjar.disabled] controls whether the target \texttt{unjar} is executed or not.\par By default this property is unset. If it is set to an arbitrary value, the target \texttt{unjar} is not executed.
	\item[unjarexcludes] contains a comma separated list of ant patterns that specify which project specific libraries will not be contained in the project's jar file.\par The default value is empty, which means, nothing is excluded.
\end{description*}

The target \texttt{unjar} is one of the rare targets that is often overridden by specific ant scripts.
Details on that can be found in section \ref{sec:override}.

For actually building the output jar file, the target \texttt{jar} is used.
It creates a jar file containing all files from the class folder, all non-source files from the source folder (e.g. image files) and all files that were extracted from the target \texttt{unjar}.
It also deletes the temporary folder specified by the property \textbf{tmp.dir}.
It can be controlled by the following properties.

\begin{description*}
	\item[jar.dir] contains the relative path to the location of the project's jar file.\par The default value is \texttt{"\$\{build.dir\}/jar"}.
	\item[resource.excludes] contains a comma separated list of ant patterns that specify which resources from the source directory will not be contained in the project's jar file.\par The default value is empty, which means, nothing is excluded.
\end{description*}

All other targets that are implicitly executed by the target \texttt{build} can be found in section \ref{sec:additional}.

\subsection{Targets for cleaning}
\label{sec:cleaning}
This section introduces the targets that are used for removing generated artifacts.
There are two main targets for this purpose, \texttt{clean} and \texttt{cleanall}.
The target \texttt{clean} removes all compiled class files, all compiled test cases, all generated schema files and the temporary folder, in case it was not deleted by a failed build attempt.
The target \texttt{cleanall} removes everything that is removed by the target \texttt{clean} and additionally removes the jar file, the javadoc, the test results and calls the target \texttt{cleanall} from the ant script that builds the IST utilities.
The latter is done for convenience, so the IST utilities are never required to be rebuilt manually.

If the building of a project fails, the first step for resolving the problem is updating the projects \texttt{common}, \texttt{jgralab}, the own project and all other projects that are required by the own project.
Then the target \texttt{cleanall} should be invoked for at least the own project and \texttt{jgralab}.
Afterwards the ant script for the own project can be run normally.

The target \texttt{clean} calls the target \texttt{deleteGeneratedSchemaFiles}, which is responsible for deleting the Java files that have been generated by the target \texttt{generateschema}.
It is only invoked, if the property \textbf{schema.file} is set.
%The path information is taken from the tg file using JGraLab.

The target \texttt{cleanall} depends on the the target \texttt{deleteConvertedSchemaFile}, which deletes the tg file, specified by the property \textbf{schema.file}, if the property \textbf{xmi.schema.file} is set.


\subsection{Additional targets}
\label{sec:additional}
This section introduces all remaining targets.
Most of them are not implicitly called by \texttt{build}.

\subsubsection{Implicitly called targets}
The target \texttt{createClassesDir} only creates the directory for the class files specified by the property \textbf{classes.dir}.

The target \texttt{customAntTasks} depends on the target \texttt{jgralab}.
It ensures that JGraLab's jar file is built and defines the ant tasks \texttt{tgschema2java}, \texttt{deletegeneratedschema} and \texttt{rsa2tg}.
These ant tasks are implemented in JGraLab.
All targets that use these tasks have to depend on \texttt{customAntTasks} rather than depend directly on \texttt{jgralab}.
This can be important when overriding targets.

\subsubsection{The target \texttt{ensureJarExists}}
The target \texttt{ensureJarExists} only invokes the target \texttt{build}, if the project's jar file does not exist yet.
This is a very useful target for avoiding unnecessary build cycles.
However, this target cannot detect updates in the source code\footnote{This is due to the main difference of \emph{ant} in comparison to \emph{make}. \emph{ant} is target oriented, where \emph{make} is file oriented.}.

\subsubsection{The target \texttt{sourcejar}}
The target \texttt{sourcejar} creates a second jar file, based on the normal one.
It makes a copy of the normal jar file and adds all Java source files from the source folder to it.
It depends on the target \texttt{ensureJarExists}.

\subsubsection{The target \texttt{document}}
The target \texttt{document} creates the javadoc for the project.
It can be controlled by the following properties.
\begin{description*}
	\item[doc.dir] contains the relative path to the location of the project's javadoc files.\par The default value is \texttt{"\$\{build.dir\}/doc"}.
	\item[documentexcludes] contains a list of ant patterns that specify which classes are not being added to the javadoc.\par The default value is empty, which means, nothing is excluded.
	\item[document.access] contains the information down to which visibility level the javadoc is generated.\par The default value is \texttt{"public"}. Valid values are \texttt{"public"}, \texttt{"protected"}, \texttt{"package"} and \texttt{"private"}.
\end{description*}

%The javadoc is generated into the directory specified by the property \textbf{doc.dir}.

\subsubsection{The target \texttt{run}}
The target \texttt{run} invokes the main method of the main class.
The main class is defined by the property \textbf{main.fq}.
The target can be controlled by the following properties.

\begin{description*}
	\item[run.args] contains the arguments that are passed to the main method.\par The default value is empty.
	\item[run.jvmargs] contains the arguments that are passed to the jvm.\par The default value is empty.
	\item[run.dir] contains a reference to the directory the application is executed from.\par The default value is set to \texttt{"\$\{project.dir\}"}.
\end{description*}

\subsubsection{The target \texttt{test}}
The target \texttt{test} compiles the test cases and runs JUnit tests.
%It assumes the test cases to be in the directory specified by the property \textbf{testcases.dir}, compiles the test cases to the directory specified by the property \textbf{testclasses.dir} and writes the test results to the directory specified by the property \textbf{testresults.dir}.
It can be controlled by the following properties.
\begin{description*}
	\item[testcases.dir] contains the relative path to the source files of the junit test cases.\par The default value is \texttt{"\$\{project.dir\}/testit"}.
	\item[testclasses.dir] contains the relative path to the compiled junit test cases.\par The default value is \texttt{"\$\{build.dir\}/testclasses"}.
	\item[testresults.dir] contains the relative path to the directory that should contain the test results.\par The default value is \texttt{"\$\{build.dir\}/testresults"}.
	\item[test.suite] contains the fully qualified name of the test suite to be executed.\par By default this property is unset. Only if it is set to a value, the target \texttt{test} is actually executed.
	\item[test.formattertype] contains a string representing the style of the test results.\par The default value is set to \texttt{"brief"}. Valid values are \texttt{"brief"}, \texttt{"plain"}, \texttt{"xml"} and \texttt{"failure"}.
\end{description*}

\subsubsection{The target \texttt{modify}}
The target \texttt{modify} changes the build id of the project in the specified main class.
It can be controlled by the following properties.

\begin{description*}
	\item[main.src] contains a reference to the path of the source file of the main class.\par The default value is set to \texttt{"\$\{src.dir\}/\$\{basePackagePath\}/\$\{main\}.java"}.
\end{description*}

\subsubsection{The target \texttt{addLicenseHeaders}}
The target \texttt{addLicenseHeaders} adds a license header to every Java source file in the source folder.
Existing headers will be replaced by the new one.
It depends on the target \texttt{clean} for avoiding setting a header to the generated schema source files.
It can be controlled by the following properties.
\begin{description*}
	\item[license.file] contains a reference to the path of the file containing the license header that will be added to all Java files.\par By default this property is unset. Only if it is set to a value, the target \texttt{addLicenseHeaders} is actually executed.
\end{description*}

\section{Further customizations of specific ant scripts}
\label{sec:customize}
Section \ref{sec:properties} described how the custom ant script can be customized by overriding properties.
Section \ref{sec:classpath} showed how the classpath can be extended.
Both of these features are sufficient for many cases.
However, if a project is more complex or has more complex dependencies, it is most likely required to further customize its specific ant script.
This section shows where such customization might be required and describes the conventions that should be obeyed.

\subsection{Customizing targets}
Any specific ant script, that imports the common ant script, does not only include its source code, as section \ref{sec:properties} might indicate.
This import act also includes an inheritance relation between the common ant script and the specific ant script.
This particularly concerns the targets defined in the common ant script.

The targets inherited from the common ant script can be overridden.
Furthermore an analogy to the super call in Java exists.
Also a mechanism similar to the polymorphic method call in Java can be found.

The details about these features and how they can be used for customizing a specific ant script are described in the following.

It can be feasible to add targets to a specific ant script.
These targets can be used for creating callable targets that handle additional jobs which are not concerned to the build process (e.g. the target \texttt{release} in JGraLab).
They can also be used for refining and overriding inherited targets.

\subsubsection{Overriding inherited targets}
\label{sec:override}
A target can be overridden by just creating a target with the same name in the specific ant script.
This has two effects.
The obvious effect is, if the target is called directly or indirectly by dependency in the specific ant script, the overridden variant is invoked.
The not so obvious effect is, if a target in the common ant script depends on the target that has been overridden in the specific ant script, also the overridden variant is invoked.
This behavior is comparable to polymorphic method calls in Java and allows a very flexible way of manipulating the behavior of the specific ant script.

Listing \ref{lst:common_build} shows the build target of the common ant script.
\begin{lstlisting}[caption=Target built in the common ant script,label=lst:common_build,float=!ht,language=ant]
	<target name="build" depends="jgralab,clean,compile,jar" />
\end{lstlisting}

Assuming the project of a specific ant script has defined a new target \texttt{pre\_compile} that should be invoked before the target \texttt{compile} is invoked.
For achieving this, the build target can be overridden as shown in listing \ref{lst:specific_build}.

\begin{lstlisting}[caption=Target built in a specific ant script,label=lst:specific_build,float=!ht,language=ant]
	<target name="build" depends="jgralab,clean,pre_compile,compile,jar" />
\end{lstlisting}

Here the build target is overridden and only the dependency has changed.
This solution has one problem.
If the target \texttt{compile} is called directly, the target \texttt{pre\_compile} is not invoked.
For this purpose, it would be feasible to let \texttt{compile} depend on \texttt{pre\_compile}.
For achieving this, the target \texttt{compile} has to be overridden.
However, the original target \texttt{compile} should also be invoked, but without copying its content from the common ant script.
Ant provides a mechanism to call the targets from imported ant scripts.
The property \texttt{name} from the root tag \texttt{project} is used as identifier.
This is comparable to the super reference in Java.
In the common ant script this attribute has the value \texttt{"common"}.

Listing \ref{lst:compile_override} shows how the target \texttt{compile} can be overridden with using the original target.

\begin{lstlisting}[caption=Overriding the target \texttt{compile},label=lst:compile_override,float=!ht,language=ant]
	<target name="compile" depends="pre_compile,common.compile" />
\end{lstlisting}

Please note that the order of the required targets in the attribute \texttt{depends} dictates the order the targets are invoked.
If the desired target can also be invoked after the inherited target, defining a new target can be avoided.
This technique can be used for overriding he target \texttt{unjar} because the order of the unpacked libraries is not important.

The following example assumes the common library for command line parsing (Apache commons CLI) is required in the project\footnote{which is true for all projects including a main class with command line interface}.
For creating a standalone jar file, adding this library to the classpath, as described in section \ref{sec:classpath} is not sufficient.
Its content has to be added to the project's jar as well.
The common target \texttt{unjar} does not consider libraries from \texttt{classpathExtension}.
So the target \texttt{unjar} should be extended for adding this behavior.

Listing \ref{lst:unjar_override} shows how this is done.

\begin{lstlisting}[caption=Overriding the target \texttt{unjar},label=lst:unjar_override,float=!ht,language=ant]
	<target name="unjar" depends="common.unjar">
		<unjar dest="${tmp.dir}">
			<fileset dir="${comlib.dir}/apache/commons/cli/" includes="**/*.jar" />
		</unjar>
	</target>
\end{lstlisting}

The target depends on the target \texttt{unjar} from the common ant script (line 1).
The unjar task (lines 2 - 4) unpacks all jar files that are found in the specified location into the temporary directory.
The jar files are refered to in analogy to extending classpaths (see section \ref{classpath}.

\subsubsection{Convention for target dependencies}
\label{sec:target:conventions}
This section shows some conventions for the dependencies of targets.

The common ant script is designed to build a project, when the target \texttt{build} is called.
The target \texttt{clean} should only clean the generated class files and schema files, but not the jar file, the test results and the documentation.
This behavior should never be changed.

All targets that are invoked by \texttt{build} should also work when invoked directly.
However, it might be better if they do not depend on each other.
E.g. the target \texttt{jar} does not depend on \texttt{compile}.
Assuming a big project that requires several minutes for compilation.
If a resource is changed (e.g. an image file), but the source code remains unchanged, it is better if the jar target only rebuilds the jar file using the class files that are already compiled instead of rebuilding the whole project.




\subsection{Depend a project on other projects}
\label{sec:dependency}

\clearpage
 
%\addcontentsline{toc}{section}{References}
%\bibliographystyle{geralpha}
%\bibliographystyle{ieeetr}
%\bibliography{references}

\end{document}
