% document class options
%----------------------------------------------------------------------------------------------
\documentclass
[
		a4paper,
		twoside, 										
		BCOR10mm,											
		11pt,												
		halfparskip,								
		bigheadings,								
		notitlepage,			
		pdftex											
]
{scrartcl}												


% used packages
%----------------------------------------------------------------------------------------------
\usepackage[utf8]{inputenc}
\usepackage{listings}
\usepackage[T1]{fontenc}				
\usepackage{palatino}							
\usepackage{url}									
\usepackage{paralist}							


% other 
%----------------------------------------------------------------------------------------------
\sloppy


% document options
%----------------------------------------------------------------------------------------------
\begin{document}

\pagestyle{headings}							

\title{Common Apache ANT file}									
\author{Kerstin Falkowski \\ falke@uni-koblenz.de}
\date{Version 0.1, 20.09.2007}			
\maketitle

\tableofcontents

%-----------------------------------------------------------------
\section{Common Apache ANT file}
\label{CommonApacheANTFile}

\textbf{Apache ANT} is 'a software tool for \emph{automating software build processes}. It is similar to \emph{make} but is written in the \emph{Java} language, requires the Java platform, and is best suited to building Java projects.'\footnote{\url{http://en.wikipedia.org/wiki/Apache_Ant}}

Disclaimer: This is no tutorial for Apache ANT itself. For further information see the Apache ANT Website\footnote{\url{http://ant.apache.org}}\footnote{\url{http://ant.apache.org/manual}}.

We created an ANT file containing a couple of \textbf{targets that are required in most of our projects depending on JGraLab}. This document describes how one can use this common ANT file in such a project. Section  \ref{ProjectRequirements} describes the project requirements that have to be fulfilled in order to use the common ANT file and how to create such a project automatically. Section \ref{CommonANTFileDescription} describes the realized targets in the common ANT file and how they can be used in local ANT files. Section \ref{ExampleLocalANTFileDescription} describes a minimal local ANT file, its dependencies and common ANT file calls.

The following projects are already using the common ANT file:
\begin{compactitem}
	\item jgralab
	\item rsleditor
\end{compactitem}
Attention: add your project to the list if it uses the ANT file too!


%-----------------------------------------------------------------
\subsection{Project Requirements}
\label{ProjectRequirements}

There are some requirements that have to be fulfilled by a project in order to use the common ANT file.


%-----------------------------------------------------------------
\subsubsection{Folder structure}
\label{FolderStructure}

The project folder must be a child of the \texttt{project} folder in your sandbox. The absolute location of the sandbox doesn't matter, here \texttt{...re-group} is the base directory of your checkout of the SVN repository \texttt{https://svn.uni-koblenz.de/gup/re-group}.

\footnotesize
\begin{compactitem}
	\item \texttt{...re-group/project/\$projectname}
\end{compactitem}
\normalsize

The following folders and their content have to be inside:
\footnotesize
\begin{compactitem}
	\item \texttt{lib} (contains project specific libraries)
	\item \texttt{src/de/uni\_koblenz/\$projectname} (contains java source code, schema files and grammar files)
	\item \texttt{testit/de/uni\_koblenz/\$projectnametest} (contains junit test cases)
\end{compactitem}
\normalsize

The following folders are generated as well as deleted by the ant-file and therefore should not be used otherwise:
\footnotesize
\begin{compactitem}
	\item \texttt{build/classes}
	\item \texttt{build/doc/html}
	\item \texttt{build/jar}
	\item \texttt{build/temp}
	\item \texttt{build/testresults}
\end{compactitem}
\normalsize


%-----------------------------------------------------------------
\subsubsection{Adapt an exisiting project}
\label{LocalANTFile}

If your project already exists and has got the required folder structure (section \ref{FolderStructure}) you can copy \texttt{regroup/project/common/newproject/build.xml} into your project's top folder and rename the value of the property \texttt{projectname} inside the file as \texttt{\$projectname}. 


%-----------------------------------------------------------------
\subsubsection{Creation of a new project}
\label{CreationOfANewSuitableProject}

If you want to create a new project see section \ref{Newproject}.



%-----------------------------------------------------------------
\subsection{Common ANT file description}
\label{CommonANTFileDescription}

The common ANT file contains the targets listet below.


%-----------------------------------------------------------------
\subsubsection{Build}
\label{Build}

The target \textbf{build} does nothing.


%-----------------------------------------------------------------
\subsubsection{Modify}
\label{Modify}

The target \textbf{modify} modifies the build-id of a JGraLab project.

The build-id of JGraLab itself is stored in its main-class \texttt{de.uni\_koblenz.jgralab.JGraLab}.
This class provides meta-information and prints it to the console. This information includes the build-id and the revision-id. The latter is automatically generated by svn. Svn generates this value, if the content of the file has changed.

To force a change of the file's content, the build-id is used. The target \emph{modify} uses a small tool, that performs this task automatically. After that, a java-based svn-client is called to check in the modified file. Svn changes the revision-id during this process.

This mechanism can be used in other JGraLab-projects, to keep the revision id up-to-date.
You have to include the following code in the main-class of your project:

\begin{lstlisting}[language = java]
 private final String revision = "$Revision: 1234 $";
 private final String buildID = "1";
\end{lstlisting}

The build-id will be incremented each time this target is called. Additionally, the Revision-String is kept up to date, because of the svn-checkin this task performs. If this checkin fails, the Revision is not updated.


%-----------------------------------------------------------------
\subsubsection{Clean}
\label{Clean}

The target \textbf{clean} deletes all classes, the jar, the temporary files and all generated schema and parser files by deleting the following folders:
\footnotesize
\begin{compactitem}
	\item \texttt{build/classes}
	\item \texttt{build/jar}
	\item \texttt{build/temp}	
	\item \texttt{src/de/uni\_koblenz/<subprojectname>/schema}
	\item \texttt{src/de/uni\_koblenz/<subprojectname>/parser}
\end{compactitem}
\normalsize

\textbf{clean} should be called from a local ANT file using the following command: 
\footnotesize
\verb|<ant dir="${common.dir}" antfile="build.xml" target="clean"/>|
\normalsize.

If your schema and/or parser files are from a subproject of your project, you can give a \texttt{\$subprojectname} referring to your projectname:\\
\footnotesize
\verb|<ant dir="${common.dir}" antfile="build.xml" target="clean">|\\ \verb|   <property name="subprojectname" value="$subprojectname>"/>|\\ \verb|</ant>|
\normalsize\\
(see JGraLab for example)


%-----------------------------------------------------------------
\subsubsection{Cleanall}
\label{Cleanall}

The target \textbf{cleanall} deletes all classes, the documentation, the jar, the temporary files, the testresults and all generated schema and parser files by deleting the following folders:
\footnotesize
\begin{compactitem}
	\item \texttt{build}
	\item \texttt{src/de/uni\_koblenz/<subprojectname>/schema}
	\item \texttt{src/de/uni\_koblenz/<subprojectname>/parser}
\end{compactitem}
\normalsize

\textbf{cleanall} should be called from a local ANT file using the following command: 
\footnotesize
\verb|<ant dir="${common.dir}" antfile="build.xml" target="cleanall"/>|
\normalsize.

If your schema and/or parser files are from a subproject of your project, you can give a \texttt{\$subprojectname} referring to your projectname:\\
\footnotesize
\verb|<ant dir="${common.dir}" antfile="build.xml" target="cleanall">|\\ \verb|   <property name="subprojectname" value="$subprojectname"/>|\\ \verb|</ant>|
\normalsize\\
(see JGraLab for example)


%-----------------------------------------------------------------
\subsubsection{Compile}
\label{Compile}

The target \textbf{compile} compiles all Java sources in:
\footnotesize
\begin{compactitem}
	\item \texttt{src}
\end{compactitem}
\normalsize
and puts the generated files into:
\footnotesize
\begin{compactitem}
	\item \texttt{build/classes}
\end{compactitem}
\normalsize

\textbf{compile} should be called from a local ANT file using the following command: 
\footnotesize
\verb|<ant dir="${common.dir}" antfile="build.xml" target="compile"/>|
\normalsize.

If you want to include or exclude some directories or files you can give \texttt{\$compileincludes} and \texttt{\$compileexcludes} referring to the source folder:\\
\footnotesize
\verb|<ant dir="${common.dir}" antfile="build.xml" target="compile"/>|\\ \verb|   <property name="compileincludes" value="$compileincludes"/>|\\ \verb|   <property name="compileexcludes" value="$compileexcludes"/>|\\ \verb|</ant>|
\normalsize\\
(see JGraLab for example)


%-----------------------------------------------------------------
\subsubsection{Jarfiletest}
\label{Jarfiletest}

The target \textbf{jarfiletest} tests the existence of a project's jar file in
\footnotesize
\begin{compactitem}
	\item \texttt{build/jar}
\end{compactitem}
\normalsize
and creates it if it does not exist at that time.

\textbf{jarfiletest} should be called from a local ANT file using the following command: 
\footnotesize
\verb|<ant dir="${common.dir}" antfile="build.xml" target="jarfiletest"/>|
\normalsize.

If you want test the existence of another projects jar file than the calling one, you can give the other projects \texttt{\$projectname}:\\
\footnotesize
\verb|<ant dir="${common.dir}" antfile="build.xml" target="jarfiletest">|\\ \verb|   <property name="projectname" value="$projectname"/>"/>|\\ \verb|</ant>|
\normalsize\\
(see RSLEditor for example)


%-----------------------------------------------------------------
\subsubsection{Generateschema}
\label{Generateschema}

The target \textbf{generateschema} generates JGraLab Java schema sources using a given \texttt{\$schemafile} (file extension \texttt{.tg}) and the JGraLab utility \texttt{TgSchema2Java} and puts them into:
\footnotesize
\begin{compactitem}
	\item \texttt{src/de/uni\_koblenz/<subprojectname>/schema}
\end{compactitem}
\normalsize

\textbf{generateschema} should be called from a local ANT file using the following command:\\
\footnotesize
\verb|<ant dir="${common.dir}" antfile="build.xml" target="generateschema">|\\ \verb|   <property name="schemafile" value = "$schemafile.tg"/>"/>|\\ \verb|</ant>|
\normalsize

If your schema files are from a subproject of your project, you can also give a \texttt{\$subprojectname} referring to your projectname:\\
\footnotesize
\verb|<ant dir="${common.dir}" antfile="build.xml" target="generateschema">|\\ \verb|   <property name="subprojectname" value = "$subprojectname"/>|\ \verb|   <property name="schemafile" value = "$schemafile.tg"/>"/>|\\ \verb|</ant>|
\normalsize
\\(see JGraLab for example)


%-----------------------------------------------------------------
\subsubsection{Generateparser}
\label{Generateparser}

The target \textbf{generateparser} generates parser sources using a given \texttt{\$grammarfile} (file extension \texttt{.g}) and \texttt{ANTLR} and puts them into:
\footnotesize
\begin{compactitem}
	\item \texttt{src/de/uni\_koblenz/<subprojectname>/parser}
\end{compactitem}
\normalsize

\textbf{generateparser} should be called from a local ANT file using the following command:\\
\footnotesize
\verb|<ant dir="${common.dir}" antfile="build.xml" target="generateparser">|\\ \verb|   <property name="schemafile" value = "$parserfile.tg"/>"/>|\\ \verb|</ant>|
\normalsize

If your parser files are from a subproject of your project, you can also give a \texttt{\$subprojectname} referring to your projectname:\\
\footnotesize
\verb|<ant dir="${common.dir}" antfile="build.xml" target="generateparser">|\\ \verb|   <property name="subprojectname" value = "$subprojectname"/>|\ \verb|   <property name="parserfile" value = "$parserfile.tg"/>"/>|\\ \verb|</ant>|
\normalsize
\\(see JGraLab for example)


%-----------------------------------------------------------------
\subsubsection{Unjar}
\label{Unjar}

The target \textbf{unjar} unpacks all libs insde:
\footnotesize
\begin{compactitem}
	\item \texttt{lib}
\end{compactitem}
\normalsize
and puts the unpacked source files into:
\footnotesize
\begin{compactitem}
	\item \texttt{build/temp}
\end{compactitem}
\normalsize

\textbf{unjar} should be called from a local ANT file using the following command: 
\footnotesize
\verb|<ant dir="${common.dir}" antfile="build.xml" target="unjar"/>|
\normalsize


%-----------------------------------------------------------------
\subsubsection{Jar}
\label{Jar}

The target \textbf{jar} creates an executable jar file containing all files in:
\footnotesize
\begin{compactitem}
	\item \texttt{build/classes}
	\item \texttt{build/temp}
\end{compactitem}
\normalsize
and puts the jarfile into:
\footnotesize
\begin{compactitem}
	\item \texttt{build/jar}
\end{compactitem}
\normalsize

\textbf{jar} should be called from a local ANT file using the following command: 
\footnotesize
\verb|<ant dir="${common.dir}" antfile="build.xml" target="jar"/>|
\normalsize


%-----------------------------------------------------------------
\subsubsection{Sourcejar}
\label{Sourcejar}

The target \textbf{sourcejar} creates a not executable jar file containing all files in:
\footnotesize
\begin{compactitem}
	\item \texttt{src}
\end{compactitem}
\normalsize
and puts the jarfile into:
\footnotesize
\begin{compactitem}
	\item \texttt{build/jar}
\end{compactitem}
\normalsize

\textbf{sourcejar} should be called from a local ANT file using the following command: 
\footnotesize
\verb|<ant dir="${common.dir}" antfile="build.xml" target="sourcejar"/>|
\normalsize


%-----------------------------------------------------------------
\subsubsection{Run}
\label{Run}

The target \textbf{run} tests the existence of the project jar file in:
\footnotesize
\begin{compactitem}
	\item \texttt{build/jar}
\end{compactitem}
\normalsize
creates it if it does not exist at that time and executes it.

\textbf{run} should be called from a local ANT file using the following command:\\
\footnotesize
\verb|<ant dir="${common.dir}" antfile="build.xml" target="run"/>|
\normalsize

You can also give some Arguments. To do so, set the \texttt{\$run.args} property:\\
\footnotesize
\verb|<ant dir="${common.dir}" antfile="build.xml" target="run">|\\ \verb|   <property name="run.args" value = "$run.args"/>"/>|\\ \verb|</ant>|
\normalsize
\\(see RSLEditor for example)


%-----------------------------------------------------------------
\subsubsection{Document}
\label{Document}

The target \textbf{document} documents all source files in:
\footnotesize
\begin{compactitem}
	\item \texttt{src}
\end{compactitem}
\normalsize
puts the html documentation into
\footnotesize
\begin{compactitem}
	\item \texttt{build/doc/html}
\end{compactitem}
\normalsize
and a zip-file containing the documentation into
\footnotesize
\begin{compactitem}
	\item \texttt{build/doc}
\end{compactitem}
\normalsize

\textbf{document} should be called from a local ANT file using the following command:\\
\footnotesize
\verb|<ant dir="${common.dir}" antfile="build.xml" target="document"/>|
\normalsize

If you want to exclude some directories or files you can give \texttt{\$documentexcludes} referring to the source folder:\\
\footnotesize
\verb|<ant dir="${common.dir}" antfile="build.xml" target="document">|\\ \verb|   <property name="docmentexcludes" value = "$documentexcludes"/>|\\ \verb|</ant>|
\normalsize
\\(see RSLEditor for example)


%-----------------------------------------------------------------
\subsubsection{Test}
\label{Test}

The target \textbf{test} compiles all junit test case sources in:
\footnotesize
\begin{compactitem}
	\item testit/de/uni-koblenz/\$projectnametest
\end{compactitem}
\normalsize
puts the generated classes into:
\footnotesize
\begin{compactitem}
	\item build/classes
\end{compactitem}
\normalsize
executes the testcases and puts the testresults into:
\footnotesize
\begin{compactitem}
	\item build/testresults
\end{compactitem}
\normalsize

\textbf{test} should be called from a local ANT file using the following command:\\
\footnotesize
\verb|<ant dir="${common.dir}" antfile="build.xml" target="test"/>|
\normalsize


%-----------------------------------------------------------------
\subsubsection{Newproject}
\label{Newproject}

The target \textbf{newproject} creates a new project folder \texttt{\$projectname} into the folder 
\footnotesize
\begin{compactitem}
	\item regroup/project
\end{compactitem}
\normalsize
if it does not exist at that time and prepares it for the new project. This means it creates the required project folder structure (section \ref{FolderStructure}) and copies some test sources inside.

\textbf{newproject} can not be called out of a local ANT file, it can be called only one time per project from the command line in folder
\footnotesize
\begin{compactitem}
	\item regroup/project/common
\end{compactitem}
\normalsize
using the following command:\\
\footnotesize
\verb|ant newproject -Dprojectname=$projectname|
\normalsize



%-----------------------------------------------------------------
\subsection{Example local ANT file description}
\label{ExampleLocalANTFileDescription}

A \textbf{minimal local ANT file} containing all required features for building a JGraLab depending project should contain the following targets, dependencies and calls of the common ANT file targets listet above:
\begin{compactitem}
	\item \textbf{build} (depends on \texttt{clean}, \texttt{jar})
	\item \textbf{release} (depends on \texttt{modify}, \texttt{cleanall}, \texttt{jar}, \texttt{test}, \texttt{document})
	\item \textbf{modify} (calls \texttt{modify} in common ANT file)
	\item \textbf{clean} (calls \texttt{clean} in common ANT file)
	\item \textbf{cleanall} (calls \texttt{cleanall} in common ANT file)
	\item \textbf{jgralabjarfiletest} (calls \texttt{jarfiletest} with \texttt{property projectname = jgralab} in common ANT file)
	\item \textbf{generateschema} (depends on \texttt{jgralabjarfiletest}, calls \texttt{generate\-schema} in common ANT file)
	\item \textbf{generateparser} (calls \texttt{generateparser} in common ANT file)
	\item \textbf{compile} (depends on \texttt{generateschema}, \texttt{generateparser}, calls \texttt{compile} in common ANT file)
	\item \textbf{jar} (depends on \texttt{compile} calls \texttt{unjar}, \texttt{jar} in common ANT file)
	\item \textbf{run} (calls \texttt{run} in common ANT file)
	\item \textbf{test} (depends on \texttt{compile}, calls \texttt{test} in common ANT file)
	\item \textbf{document} (depends on \texttt{generateschema} and \texttt{generateparser}, calls \texttt{document} in common ANT file)
\end{compactitem}
For an example see common/newproject/build.xml or the build.xml in a new project folder generated with the common ANT file target \texttt{newproject} (section \ref{Newproject}).

\end{document}
